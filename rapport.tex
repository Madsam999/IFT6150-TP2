\documentclass{article}
% --- General Packages ---
\usepackage[utf8]{inputenc}
\usepackage[T1]{fontenc}
\usepackage[english]{babel}
\usepackage{fullpage}
\usepackage{geometry}
\usepackage{graphicx}
\usepackage{hyperref}
\usepackage{url}
\usepackage{setspace}
\usepackage{comment}
\usepackage{xcolor}

% --- Math Packages ---
\usepackage{amsmath,amssymb,amsthm,nccmath}
\usepackage{amsfonts}
\usepackage{bm}

% --- Table & Figure Packages ---
\usepackage{booktabs}
\usepackage{multirow}
\usepackage{float}

% --- Code & Text Packages ---
\usepackage{listings}
\usepackage[autostyle, english = american]{csquotes}
\usepackage{algpseudocode}
\usepackage{tikz}
\usepackage{subcaption}
\usetikzlibrary {automata,positioning}

% --- Listing Style Configuration ---
\definecolor{codegreen}{rgb}{0,0.6,0}
\definecolor{codegray}{rgb}{0.5,0.5,0.5}
\definecolor{codepurple}{rgb}{0.58,0,0.82}
\definecolor{backcolour}{rgb}{0.95,0.95,0.92}

\lstdefinestyle{mystyle}{
    backgroundcolor=\color{backcolour},
    commentstyle=\color{codegreen},
    keywordstyle=\color{magenta},
    numberstyle=\tiny\color{codegray},
    stringstyle=\color{codepurple},
    basicstyle=\ttfamily\scriptsize,
    breakatwhitespace=false,
    breaklines=true,
    captionpos=b,
    keepspaces=true,
    numbers=left,
    numbersep=1pt,
    showspaces=false,
    showstringspaces=false,
    showtabs=false,
    tabsize=2,
    inputencoding=utf8,
    extendedchars=true,
}
\setlength{\parindent}{0in}
\setlength{\parindent}{0in}
\begin{document}
\begin{titlepage}
	\begin{center}
		\vspace*{1cm}

		\Huge
		\textbf{Devoir 2}

		\vspace{0.5cm}
		\LARGE

		\vspace{1.5cm}

        
		\textbf{Samuel Fournier}\\20218212 \\
		\vfill
		\textbf{Alexandre Toutant}\\20028191 \\
		\vfill


		Dans le cadre du cours\\
		IFT 6150


		\vspace{0.8cm}

		\includegraphics[width=0.4\textwidth]{udem.jpg}

		\Large
		Département d'informatique et de recherche opérationnelle\\
		Université de Montréal\\
		Canada\\
		29 octobre 2025

	\end{center}
\end{titlepage}
\section{Introduction}

Ce travail pratique avait pour but de programmer, en langage C, les principales étapes du détecteur de contours de \textbf{Canny}.  
L'objectif était de comprendre comment on peut passer d'une image en niveaux de gris à une image contenant seulement les contours importants, tout en réduisant le bruit et les faux contours.

\subsection{But du TP}
Le but du TP était de réaliser un programme capable de détecter les contours d’une image.  
Pour y arriver, le programme effectue les étapes suivantes :
\begin{enumerate}
    \item Appliquer un \textbf{flou gaussien} pour réduire le bruit;
    \item Calculer le \textbf{gradient} d’intensité avec les filtres \((-1,1)\);
    \item Trouver la \textbf{direction du gradient} (0°, 45°, 90°, 135°);
    \item Faire la \textbf{suppression des non-maximums} pour amincir les contours;
    \item Appliquer un \textbf{double seuillage} (\(\tau_L\), \(\tau_H\)) avec un suivi par hystérésis pour ne garder que les vrais bords.
\end{enumerate}

\subsection{Rôle des contours}
Les contours représentent les zones où l’intensité change brusquement.  
Ils délimitent les objets et aident à comprendre la structure d’une image.  
La détection de contours est essentielle pour plusieurs applications : segmentation d’objets, reconnaissance de formes, suivi de mouvement, etc.  
Un bon détecteur doit être à la fois précis, résistant au bruit et produire des bords fins et continus.

\subsection{Filtre gaussien}
Avant la détection des contours, un \textbf{filtre gaussien} est appliqué pour adoucir l’image et éliminer le bruit.  
Cela empêche le détecteur de confondre les petites variations d’intensité avec de faux contours.  
Dans notre code, ce filtrage est effectué dans le \textbf{domaine fréquentiel} à l’aide de la \textbf{transformée de Fourier (FFT)}.  
Le paramètre \(\sigma\) contrôle l’intensité du flou :
\begin{itemize}
    \item petit \(\sigma\) : plus de détails visibles;
    \item grand \(\sigma\) : image plus lissée, moins de bruit.
\end{itemize}
Cette étape prépare l’image pour le calcul du gradient et assure une détection de contours plus stable.

\section{Description de la méthode utilisée pour calculer le gradient, la norme et l'angle}

Le calcul du gradient permet d’identifier les zones de l’image où l’intensité varie le plus rapidement, ce qui correspond aux contours.  
Dans notre programme, le gradient est obtenu à l’aide des filtres de convolution \([-1, 1]\), appliqués séparément sur les directions horizontales et verticales.

\subsection*{Calcul du gradient}
Pour chaque pixel, deux dérivées sont calculées :
\begin{itemize}
    \item \( G_x = I(x, y+1) - I(x, y) \) pour la variation horizontale;
    \item \( G_y = I(x+1, y) - I(x, y) \) pour la variation verticale.
\end{itemize}
Ces filtres simples permettent d’estimer la direction et l’intensité du changement local dans l’image.

\subsection*{Norme du gradient}
La norme du gradient indique l'intensité du contour à chaque point :
\[
G = \sqrt{G_x^2 + G_y^2}
\]
Plus la norme est grande, plus le contour est fort.  
C’est cette valeur qui est utilisée plus tard pour la suppression des non-maximums et pour l’hystérésis.

\subsection*{Angle du gradient}
L’orientation du gradient est donnée par :
\[
\theta = \arctan{\left(\frac{G_y}{G_x}\right)}
\]
Dans le programme, cet angle est converti en degrés et ramené dans l’intervalle \([0,180]\).  
Il est ensuite estimé selon quatre directions principales : 0°, 45°, 90° et 135°, ce qui simplifie la comparaison des pixels lors de la suppression des non-maximums.

\section{Dénombrer le nombre de seuils dans le filtre de Canny}

Le filtre de Canny utilise \textbf{deux seuils} : un seuil bas (\(\tau_L\)) et un seuil haut (\(\tau_H\)).  
Le seuil haut sert à détecter les contours forts, tandis que le seuil bas garde les contours plus faibles s’ils sont reliés à un contour fort.  
Les pixels sous \(\tau_L\) sont ignorés.  
Ce double seuillage permet de conserver des bords continus tout en réduisant les faux contours causés par le bruit.

\section{Description de la méthode utilisée pour calculer les seuils à partir de l'histogramme}

\textbf{Sam }

\section{Description de l’approximation de l’angle}

L’angle du gradient est calculé à l’aide de la fonction \(\arctan\left(\frac{G_y}{G_x}\right)\), ce qui donne une valeur comprise entre \(0\) et \(180\) degrés.  
Pour simplifier le traitement lors de la suppression des non-maximums, on ne garde pas l’angle exact.  
On retourne plutôt la valeur parmi \(0^\circ\), \(45^\circ\), \(90^\circ\) ou \(135^\circ\) qui est la plus proche de l’angle réel calculé.  
Cela permet de déterminer plus facilement la direction du contour sans perdre d’information importante.

\section{Description du but de la suppression des non-maximums dans le filtre de Canny}
\section{Description du but du seuillage pas hystérisis dans le filtre de Canny}
\section{Présentation des résultats expérimentaux}
\section{Discussion et conclusion}

\end{document}